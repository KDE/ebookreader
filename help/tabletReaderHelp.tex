\documentclass[journal,12pt]{IEEEtran}

\usepackage[utf8x]{inputenc}
\usepackage{url}

\title{TabletReader User Manual}
\author{Bogdan Cristea\\\textit{e-mail: cristeab@gmail.com}}
\date{\today}

\begin{document}

\maketitle

\section{Introduction}

\IEEEPARstart{T}{}abletReader has been designed to be a simple e-book reader for touch-enabled devices (e.g. tablets, ultrabooks). Okular core library is used as backend (\url{http://okular.kde.org/}), so that all okular supported e-book formats are available (PDF, DJVU, CHM, etc.). TabletReader can be used on devices without a touch screen, but in this case a keyboard and a mouse are needed. TabletReader needs a screen large enough (at least 800x600 pixels). Devices with screens of smaller size (e.g. smartphones) may not be suitable for reading e-books using A4 or larger page sizes since the displayed text does not adapt to the screen size (as Acrobat Reader does).

Through its interface, tabletReader enables to open e-books found on the local filesystem, to bookmark an open document in a favorites list, to switch between full screen and normal screen modes, to go to a specific page, to zoom the displayed document, to display document properties, to show this help, to display information about this software and to exit from the application. At exit, the current e-book document path and name, the current page and the zoom factor are saved so that the next time you launch tabletReader the document is opened at the same page you previously were. 

The application interface adapts to the current language of the operating system, so that English is the default language, but French and Romanian are also supported.

This software has been intended to be used on Linux-based operating systems with a graphical user interface (e.g. Active Plasma \url{http://plasma-active.org/}), since there seems to be a lack of e-book readers for touch-enabled devices runnning Linux. However Windows is also supported and since this software has been written using Qt application framework, it can be ported to other platforms provided that okular core library is available (see implementation details section below).

\newpage

\section{Utilisation}

\subsection{Moving on the same page}
\begin{itemize}
 \item using vertical or horizontal swipes (you could try this on the current document) or
 \item using the mouse wheel (if available)
\end{itemize}

\subsection{Switching between document pages}
\begin{itemize}
 \item pressing the left or right side of the main window (you could try this on the current document) or
 \item using left or right arrow key
\end{itemize}

\subsection{Toolbar buttons}

By default the toolbar is not shown. In order to access the toolbar one needs either to press for more than $2$ seconds or to double click/tap the upper side of the main window.

\begin{itemize}
 \item \textit{Open}: opens an e-book found on the local filesystem using the file browser. Vertical swipes (or the mouse wheel, if available) can be used to display the content of the current folder. The first item of the list allows to go up one level, while selecting a folder item allows to go down one level. Only suported file types are shown, however if the file cannot be opened, an error message is displayed. The last item in the file list allows to close the file browser without selecting a file.
 \item \textit{Favorites}: displays a list with favorite e-books. The first entry in the list allows to bookmark the current document at the current page. Previously bookmarked e-books can be opened by selecting the corresponding entry in the favorites list. The last entry in the list allows to close the favorites browser without selecting an entry. Note that the maximum number of favorite e-books is limited to $10$ so that when this limit is reached the oldest entry is removed. Also, a single bookmark is allowed per e-book, when the same   e-book is bookmarked again, the previous entry is removed.
 \item \textit{Full Screen}: the application will occupy the entire screen. In full screen mode this button changes to ''Normal Screen`` and allows to restore the previous size of the main window.
 \item \textit{Go To Page}: displays a numerical keypad allowing to type the desired page number. Note that by pressing OK without typing some page number or if the page number is invalid the numerical key will dissapear and the page will not be changed.
 \item \textit{Zoom}: displays a dialog allowing to select the desired zoom factor. The default zoom factor is fit width. The zoom factor can also be changed by using pinch gestures.
 \item \textit{Properties}: displays document and application properties: document path, current page and the number of pages, elapsed time, battery status (if QtMobility is available).
 \item \textit{Help}: displays this help manual. If previously another document has been opened, the Help button is replaced by a Back button so that the user can go back to the previously opened document.
 \item \textit{About}: displays a dialog with information about this software: version number, supported e-book formats.
 \item \textit{Quit}: quits the application and saves into a local file the current document path and name, current page and the  zoom factor so that the next time when the application is started the saved settings will be used. Thus you could continue to read the document from the page you closed the application the last time.
\end{itemize}

\section{Implementation details}
TabletReader has been written in C++ using Qt application framework, QML for the graphical user interface and okular core library for handling e-book formats. Note that okular core library depends on KDE framework and other 3rd party libraries (e.g. poppler library) which need to be available too (see for further details \url{http://okular.kde.org/}).

The application is build around a circular buffer holding three pages which allows to display the e-book pages using a PathView element. In general, the circular buffer will hold the currently displayed page, the previous page and the next one so that when the user switches between adjacent pages this is done as fast as possible. 

When loading a new document, only the current page is synchronously loaded, while the previous and the next pages are loaded asynchronously in the background. Similarly, when changing between adjacent pages, the page from the circular buffer is displayed and then a new page is loaded asynchronously so that, with respect to the currently displayed page, the previous and the next page are always ready to be displayed in the circular buffer. Due to implementation details, the circular buffer needs to be filled before the user is allowed to switch the page. When the circular buffer is not ready and the user tries to switch the page by pressing the right or left side of the main window an arrow is displayed instead and the page remains unchanged.

The current application settings (current document, current page, zoom factor) and the favorites list (at most $10$ entries with the document path and current page) are stored using the \textrm{QSettings} class in a portable format between operating systems. However, in order to save the current settings the application must be closed using the \textit{Quit} command. If the device is powered off while using the application the current settings are not saved because there is no way to catch the termination signal. If no settings are available, when the application is started, this help document is displayed.

The Qt internationalization support is used to adapt the interface language to the language of the operating system. English is the default language and currently translations exist for French and Romanian.

When compiled in debug mode, a log file is created in the home folder, ''tabletReader.log`` and all messages (warnings and errors) are directed to this file.

TabletReader is part of the KDE project and its sources can be found at: \url{https://projects.kde.org/projects/playground/edu/ebookreader}.

\section{License}
 Copyright \copyright 2013, Bogdan Cristea. All rights reserved.
 
 This program is free software; you can redistribute it and/or modify  it under the terms of the GNU General Public License as published by the Free Software Foundation; either version 2, or (at your option)  any later version.
 
 This program is distributed in the hope that it will be useful,  but WITHOUT ANY WARRANTY; without even the implied warranty of
 MERCHANTABILITY or FITNESS FOR A PARTICULAR PURPOSE.  See the  GNU General Public License for more details.
 
 You should have received a copy of the GNU General Public License along with this program; if not, write to the Free Software
 Foundation, Inc., 51 Franklin Street - Fifth Floor, Boston, MA 02110-1301, USA.

\end{document}
